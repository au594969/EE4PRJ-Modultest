\documentclass[a4paper,oneside,article]{memoir}


% ---------------Preamble fra Latex kursus---------------------
\usepackage[utf8]{inputenc}                 % Korrekt håndtering af æ, ø og å
\usepackage[T1]{fontenc}                    % Korrekt håndtering af æ, ø og å
\usepackage{microtype}                      % Typografisk magi! Giver bl.a. pænere orddeling
\usepackage{graphicx}                       % Gør det muligt at indsætte billeder
\usepackage{amsmath}                        % Giver adgang til uundværlige matematikting
\usepackage{siunitx}                        % Dette  gør alle mat notationer nemmer ! 
\usepackage[danish]{babel}                  % Danske betegnelser og orddeling
\renewcommand{\danishhyphenmins}{22}        % Bedre dansk orddeling

%--------------------Selv fundet Packer------------------------
\usepackage[margin=1.1in]{geometry}         % Marginer
\usepackage{ragged2e}                       % Retter alt ind til venstre
\usepackage{parskip}                        % Strækker ud til begge marginer
\usepackage{booktabs}                       % Tablesgenerator bad om det
\setlength{\parindent}{0pt}                 % Fjerner indents
%--------------------Ekstra------------------------------------
\usepackage[table,xcdraw]{xcolor}
\usepackage{floatrow}
\usepackage{url}
\usepackage[resetlabels,labeled,]{multibib}
\newcites{A}{Bilag}
%----------------------- Macroer ------------------------------


%----------------------Tabeller--------------------------------
%\toprule               Top tyklinje
%\midrule               Mid tyndlinje
%\bottomrule            Bund tyklinje

%\begin{minipage}[t]{7CM} TEXT \vspace{2mm}\end{minipage}        Tekstombrydning

%-------------------Andre nytte komandoer----------------------
%\plainbreak{x} x = antal linje jeg vil hoppe. 
%\graphicspath{ {./images/} }
