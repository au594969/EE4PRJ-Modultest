\chapter*{Test Journal}
Denne skabelon er tiltængt som lab-journal til undersøgende arbejde og er ikke tiltænkt til accepttest/modultest. \plainbreak{0.5}
\textbf{Executed by:} \plainbreak{0.5}
\textbf{Date:}



\chapter{Test Object}
Her skal testobjektet defineres klart og entydigt. Der kan evt. henvises til diagrammer i rapporten.
Det er vigtigt, at alle målepunkter er veldefinerede. 


\chapter{Objective}
Formålet er ikke at ”vise at målingerne giver samme resultat som beregnet” - dette er ikke et objektivt formål.
\plainbreak{0.5}
Husk at referere til kravsspecifikation. - Kravet skal fremgå her på skrift.

\chapter{Equipment used}
Alt væsentligt udstyr skal beskrives entydigt. Inkluder evt. tabel

\chapter{Test setup}
Som beskrevet i måleproceduren. Oftest vises her en tegning over måleopstillingen, så man klart
kan se, hvordan udstyret er tilsluttet. Her er der undtagelsesvist henvist til måleproceduren. 

\chapter{Test procedure}
Her beskrives klart og entydigt, hvordan målingen er foretaget inkl. alle ikke indlysende indstillinger af apparater. D.v.s. en fagmand M/K skal være i stand til at gentage målingen. 

\chapter{Results and Comments}
Nogle resultater kan med fordel flyttes til rapporten. Ofte angives tabeller i målejournalen og grafer i rapporten.

\chapter{Sources of error and insecurities}
Her angives væsentlige fejlkilder og usikkerheder i.f.b. med målingen. Hvis der er uoverensstemmelser mellem beregnede/simulerede og målte data, må det forklares i rapporten. Her kan
man evt. henvise til usikkerheder beskrevet i målejournalen. 
